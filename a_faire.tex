Remarque

Ceci est un code que j’aimerais définitif.

Pour l'instant il se contente de prendre en entrée une image et 9 coefficients d'une homographie et de créée trois fichier.png pour la décompostition, la "vérité terrain" et le Ripmap.
La décomposition à l'air beaucoup moins flou que les deux autres.



Voici les taches à remplir pour que la code soit "fini" :

	-regarder le zoom par zero-padding dans groundtruth (le zoom bilinéaire est opérationnel)
		pour l'instant elle crée un ringing fort visible
	
	
	
optionnel :
	-factoriser les codes qui sont actuellement redéfinit avec des noms différents (apply_homography_1pt, eq et good_modulus ont été factorisés, je [shmuel] ne sais pas s'il y en a d'autres) 
	
	-collecter tous les paramètres à un endroit (ce qu'à commencer Shmuel / autre ?) avec "parameters.h"
	
	- créer un exécutable à part (un autre main dans le dossier) qui calculerait les différences l1 et l2 entre img_dec, img_grd et img_rip ; pour l'instant c'est viho_alt.c qui calcule ces différences, donc si on veut revoir les différences/regarder la différence sur une autre sous-image, il faut tout recalculer, y compris ground truth...
	
	- si le ripmap est créé indépendamment de l'homographie, on pourrait modifier ripmap pour qui, si un fichier "nomdelinput_ripmapconstructed_ilfallaitbienunnom.png" existe déjà, il le lise plutôt que de recalculer le ripmap (c'est un des avantages du ripmap, qu'on exploite pas ici)
	
	
	
Et le "vrai" problème :
	pour l'instant ground_truth est beaucoup plus floue que decomposition et ripmap
	il faut donc réfléchir au paramètre pour comprendre
