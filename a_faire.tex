Remarque

Ceci est un code que j’aimerais définitif.

Pour l'instant il se contente de prendre en entrée une image et 9 coefficients d'une homographie et de créée trois fichier.png pour la décompostition, la "vérité terrain" et le Ripmap et les compare. On peut préciser une sous-image sur laquelle comparer.
La vérité terrain est plus floue que les deux autres, selon le sigma choisi.



Voici les taches à remplir pour que la code soit "fini" :
	toutes ces tâches ont été accomplies !
	
	
	
optionnel :
	-factoriser les codes qui sont actuellement redéfinit avec des noms différents (apply_homography_1pt, eq et good_modulus ont été factorisés, je [shmuel] ne sais pas s'il y en a d'autres)
	
	- si le ripmap est créé indépendamment de l'homographie, on pourrait modifier ripmap pour qui, si un fichier "nomdelinput_ripmapconstructed_ilfallaitbienunnom.png" existe déjà, il le lise plutôt que de recalculer le ripmap (c'est un des avantages du ripmap, qu'on exploite pas ici)
	
	
	
Et le "vrai" problème :
	pour l'instant ground_truth est plus floue que decomposition et ripmap
	il faut donc réfléchir au paramètre pour comprendre

problèmes de bords :
	le rip map semble périodiser d'un pixel sur un bord (bien visible avec ./viho_alt lena.png 3.865385 0 -763.076923 1.579670 2.285714 -763.076823 0.011447 0 -1.289377 230 190 440 310), la vérité terrain symétrise peut-être d'un demi pixel sur le bord, la décomposition symétrise d'un pixel sur le bord
	du coup aucun bord n'est au même endroit, ce qui se voit visuellement. Ce n'est pas important tant qu'on calcule l'erreur sur des sous-images bien choisies
